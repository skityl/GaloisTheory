\documentclass{unswmaths}
\usepackage{unswshortcuts}
\begin{document}
\subject{Galois Theory}
\author{Edward McDonald}
\title{Assignment 2}
\studentno{3375335}
\setlength\parindent{0pt}
\newcommand{\Intgr}{\mathbb{Z}}
\newcommand{\Aut}{\operatorname{Aut}}
\newcommand{\isom}{\sim}
\newcommand{\Rat}{\mathbb{Q}}
\newcommand{\id}{\operatorname{id}}

\unswtitle{}

\section*{Question 1}

For this question we work over the field $\Rat$,
and $f(x) := x^3+2x+2$.
\begin{lemma}
    $f$ has no roots in $\Rat$.
\end{lemma}
\begin{proof}
    Let $a,b \in \Intgr$ with $f(a/b) = 0$ and $\gcd(a,b) = 1$. Then
    \begin{equation*}
        a^3+2ab^3+2b^3 = 0.
    \end{equation*}
    Hence, $2|a^2$. Since $2$ is prime, we conclude that $2|a$.
    
    Let $a = 2c$, then
    \begin{equation*}
        8c^3+4cb^3+2b^3 = 0.
    \end{equation*}
    Hence $4|2b^3$, so $2|b^3$. Thus, $2|b$.
    
    This contradicts $\gcd(a,b) = 1$. Hence $f$
    has no rational roots.    
\end{proof}

\begin{lemma}
    $f$ is irreducible in $\Rat[x]$.
\end{lemma}
\begin{proof}
    If $f = gh$ for $g,h\in \Rat[x]$, then $\deg(g)+\deg(h) = 3$.
    So without loss of generality $\deg(g) = 1$. But this means
    $f$ has a rational root, which is impossible.    
\end{proof}
\begin{lemma}
    $f$ has exactly one root over $\Rl$.
\end{lemma}
\begin{proof}
    We compute $f'(x) = 3x^2+2 \geq 2 > 0$. Hence 
    the function $f$ is montonically everywhere increasing.
    
    Since $f(-1) = -1$ and $f(0) = 2$, by the intermediate value
    theorem there is some $c \in (-1,0)$ such that $f(c) = 0$.
    Since $f$ is monotonically increasing, this is the unique
    zero of $f$ over $\Rl$.
\end{proof}
Now let $\beta_1,\beta_2,\beta_3 \in \Cplx$ be the distinct
roots of $f$ over $\Cplx$, with $\beta_2 = \overline{\beta_3}$,
and $\beta_1 \in \Rl$. Let $K = \Rat(\beta_1,\beta_2,\beta_3)$.
\begin{lemma}
    Let $\sigma:\Cplx\rightarrow \Cplx$ be the complex
    conjugation function, $\sigma(z) = \overline{z}$. Then
    $\sigma \in \Aut(K/\Rat)$, $\sigma^2 = \id_{\Cplx}$
    and $\sigma(\beta_2) = \beta_3$.
\end{lemma}
\begin{proof}
    It is evident that $\sigma \in \Aut(\Cplx)$,
    and $\sigma(z) = z$
    for all $z \in \Rat$ and $\sigma^2 = \id_{\Cplx}$.
    
    By definition, $\sigma(\beta_2) = \beta_3$,
    
    Let $z \in K$. Then there is some $p \in \Rat[x,y,z]$
    such that $z = p(\beta_1,\beta_2,\beta_3)$.
    Since $\sigma$ fixes $\Rl$,
    we have $\sigma(z) = p(\beta_1,\beta_3,\beta_2) \in K$.  
    
    Hence $\sigma \in \Aut(K/\Rat)$ has order 2.
\end{proof}
\begin{lemma}
    As a $\Rat$-space, $\Rat(\beta_1)$ has a basis $\{1,\beta_1,\beta_1^2\}$,
    and $[\Rat(\beta_1):\Rat] = 3$.
\end{lemma}
\begin{proof}
    Any $x \in \Rat(\beta_1)$
    can be expressed as a polynomial,
    \begin{equation*}
        x = b_0+b_1\beta_1+b_2\beta_1^2+b_3\beta_1^3+\cdots
    \end{equation*}
    with coefficients $b_j \in \Rat$.
    However since $\beta_1^3 = -2-2\beta_1$,
    we can ignore terms of order greater than $2$.
    
    Hence $\{1,\beta_1,\beta_1^2\}$
    spans $\Rat(\beta_1)$.
    
    These elements of $K$ are linearly independent over $\Rat$,
    since otherwise $\beta_1$ would satisfy some quadratic
    in $\Rat[x]$, which this is impossible as $f$ is irreducible.
    
    Hence $\Rat(\beta_1)$ is three dimensional as a $\Rat$-space.
    Thus $[\Rat(\beta_1):\Rat] = 3$.
\end{proof}


It is clear that if $\sigma \in \Aut(K/\Rat)$, and $f(\gamma) = 0$
then $f(\sigma(\gamma)) = 0$.
\begin{theorem}
\label{injection}
    There is an injective group homomophism, $\Aut(K/\Rat)\rightarrow S_3$.
\end{theorem}
\begin{proof}
    Let $\sigma \in \Aut(K/\Rat)$. Then for $j \in \{1,2,3\}$,
    $\sigma(\beta_j) = \beta_{\tau_\sigma{j}}$ for some $\tau_\sigma \in S_3$.
    Since $K$ is generated by $\{\beta_1,\beta_2,\beta_3\}$
    as a $\Rat$-algebra, $\sigma$ is uniquely determined by its
    values on $\{\beta_1,\beta_2,\beta_3\}$.
    
    Denote that map $\psi:\Aut(K/\Rat)\rightarrow S_3$
    by $\psi(\sigma) = \tau_\sigma$.
    
    Let $\sigma_1,\sigma_2 \in \Aut(K/\Rat)$. 
    
    Then $(\sigma_1\circ \sigma_2)(\beta_j) = \beta_{(\tau_{\sigma_1}\circ\tau_{\sigma_2})(j)}$.
    
    Hence $\psi(\sigma_1\circ\sigma_2) = \psi(\sigma_1)\circ\psi(\sigma_2)$.
    
    Thus $\psi$ is a group homomorphism.
    
    $\psi$ must be injective, as if $\tau_{\sigma_{1}} = \tau_{\sigma_2}$, 
    then $\sigma_1(\beta_j) = \sigma_2(\beta_j)$ for all $j$.
    But elements of $\Aut(K/\Rat)$ are uniquely determined by their values
    on $\beta_1,\beta_2,\beta_3$. Hence $\sigma_1 = \sigma_2$.
\end{proof}

\begin{theorem}
    $\Aut(K/\Rat)\isom S_3$.
\end{theorem}
\begin{proof}
    Since complex conjugation is an element of order $2$
    in $\Aut(K/\Rat)$, we have a subgroup of order $2$
    so $2||\Aut(K/\Rat)|$.
    
    Since $3 = [\Rat(\beta_1):\Rat]$,
    we conclude that $3|[K:\Rat(\beta_1)][\Rat(\beta_1):\Rat] = [K:\Rat] = |\Aut(K/\Rat)|$.
    
    Thus $6 | |\Aut(K/\Rat)|$. But since there
    is an injective map from $\Aut(K/\Rat)$ to $S_3$,
    we know that $|\Aut(K/\Rat)| = 6$.
    
    Hence the map $\psi$ in theorem \ref{injection} is bijective,
    hence a group isomorphism.
\end{proof}

\section*{Question 2}
    
We let $S_0 \subset \Rl^2$ be a finite set of points, and $S_n$
is the set of points constructible from straightedge and compass
in $n$ steps.

The fields $K_n$ are defined recursively. $K_0$
is the field extension of $\Rat$ given by the coordinates of points in $S_0$
and distances between points in $S_0$, and $K_n$
is the field extension of $K_{n-1}$
generated by the coordinates of points in $S_n$
and the distances between points of $S_n$.

\begin{theorem}
    For $n \geq 1$, $K_n = K_{n-1}(\sqrt{a_1},\sqrt{a_2},\ldots,\sqrt{a_t})$
    for some $a_1,a_2,\ldots, a_t \in K_{n-1}$.
\end{theorem}
\begin{proof}
    Suppose $l_1$ and $l_2$ are lines passing through distinct points
    of $S_{n-1}$, with $l_1$ passing through $p_1,q_1 \in S_{n-1}$
    and $p_2,q_2\in S_{n-1}$. Then $l_1$ and $l_2$ can be parametrised as
    \begin{align*}
        l_1&: p_1+\lambda(q_1-p_1)\\
        l_2&: p_2+\mu(q_2-p_2)
    \end{align*}
    for parameters $\lambda \mu \in \Rl$. We can find the point of intersection
    by solving the system of linear equations
    \begin{equation*}
        p_1+\lambda(q_1-p_1) = p_2+\mu(q_2-p_2).
    \end{equation*}
    By Cramer's rule, since the coordinates of $p_1,p_2,q_1,q_2$
    are in $K_{n-1}$, the solution for $\lambda$
    and $\mu$ must also lie in $K_{n-1}$.
    
    Now suppose $C_1$ and $C_2$ are two circles with centres
    $p = (p_x,p_y),q = (q_x,q_y) \in S_{n-1}$ and radii equal to the lengths of line
    segements joining points of $S_{n-1}$. Denote the radii
    by $r_1$ and $r_2$ respectively. The cartesian equations
    for the circles are
    \begin{align*}
        C_1&: (x-p_x)^2+(y-p_y)^2 = r_1^2\\
        C_2&: (x-q_x)^2+(y-q_y)^2 = r_2^2
    \end{align*}
    We must solve this pair of equations for $x$ and $y$.
    
    Now, by the quadratic formula, $x \in K_{n-1}(\sqrt{a})$
    where $a \in K_{n-1}(y)$. Thus there is a polynomial $f \in K_{n-1}(y)[x]$
    such that
    \begin{equation*}
        (f(a)-q_x)^2+(y-q_y)^2 = r_2^2.
    \end{equation*}
    
    blah blah AG.
    
\end{proof}

\begin{lemma}
    There is an integer $s \geq 1$ such that $[K_N:K_0] = 2^s$.
\end{lemma}
\begin{proof}
    Since $K_n = K_{n-1}(\sqrt{a_1},\sqrt{a_2},\ldots,\sqrt{a_t})$, we have
    \begin{equation*}
        [K_n:K_{n-1}] = [K_{n-1}(\sqrt{a_1},\ldots,\sqrt{a_t}:K_{n-1}(\sqrt{a_2},\ldots\sqrt{a_t})]\ldots[K_{n-1}(\sqrt{a_t}):K_{n-1}].
    \end{equation*}
    Each of the terms in the product on the left is $2$, since each extension is quadratic. Hence, $[K_n:K_{n-1}]$ is a power of $2$.
    Thus,
    \begin{equation*}
        [K_N:K_0] = [K_N:K_{N-1}][K_{N-1}:K_{N-2}]\ldots[K_1:K_0]
    \end{equation*}
    is a power of $2$.
\end{proof}

\begin{theorem}
    If $S_0$ = $\{(0,0),(1,0)\}$, then no $K_n$ contains a root of $x^3-2$.
\end{theorem}
\begin{proof}
    Since the coordinates and distances in $S_0$ are rational, we have $K_0 = \Rat$. 

    Since the polynomial $x^3 - 2$ is monotonically non-decreasing over $\Rl$, it has a unique real root. By Eisenstein's criterion, 
    $x^3-2$ is irreducible, so if $\sqrt[3]{2}$ denotes the unique real root, $[\Rat(\sqrt[3]{2}):\Rat] = 3$.
    
    Now if $\sqrt[3]{2} \in K_n$ for some $n$, we have $K_n$ is a field extension of $\Rat(\sqrt[3]{2})$. 
    
    Hence, $[K_n:\Rat(\sqrt[3]{2})]$ is well defined, and
    \begin{equation*}
        [K_n:\Rat] = [K_n:\Rat(\sqrt[3]{2})][\Rat(\sqrt[3]{2}):\Rat].
    \end{equation*}
    Thus $3|[K_n:\Rat]$.
    
    But this is impossible since we know $[K_n:\Rat] = [K_n:K_0]$ is a power
    of $2$.  
    
    
\end{proof}


\end{document}
