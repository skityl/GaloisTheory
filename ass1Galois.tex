\documentclass{unswmaths}
\usepackage{unswshortcuts}
\begin{document}
\subject{Galois Theory}
\author{Edward McDonald}
\title{Assignment 1}
\studentno{3375335}
\setlength\parindent{0pt}
\newcommand{\Intgr}{\mathbb{Z}}
\newcommand{\Aut}{\operatorname{Aut}}

\unswtitle{}
For this assignment, $n > 1$ and $k$ is a field
containing a primitive $n$th root of unity $\zeta_n$,
and the characteristic of $k$ does not divide $n$.
Let $a \in k$ be such that the polynomial
$P_a(x) = x^n-a$ has no root in $k$.
Let $K_{a,n} = k(\!\sqrt[n]{a})$ be a field extention
of $k$ generated by a root of $P_a$.
\section*{Question 1}
\begin{lemma}
In $K_{a,n}$, $P_a$ is expressed as a product of $n$ distinct linear
factors
\begin{equation*}
P_a(x) = \prod_{k=1}^n (x-\zeta_n^k\sqrt[n]{a})
\end{equation*}
\end{lemma}
\begin{proof}
Since $\zeta_n$ is primitive, all
of the elements $\zeta_n^k\sqrt[n]{a}$ for $k = 1,2,\ldots,n$
are distinct and are all zeroes of $P_a$.
Hence since $P_a$ has at most $n$ roots, this is all of the roots
of $P_a$ and so $P_a$ decomposes as
\begin{equation*}
P_a(x) = \prod_{k=1}^n (x-\zeta_n^k\sqrt[n]{a})
\end{equation*}
\end{proof}
\section*{Question 2}
For this question, $\lambda \in \Intgr/n$ and
define
\begin{equation*}
\sigma_\lambda: K_{a,n}\rightarrow K_{a,n}
\end{equation*}
by $\sigma_{\lambda}(f(\sqrt[n]{a})) = f(\zeta_n^\lambda \sqrt[n]{a})$
for any polynomial $f \in k[x]$.
\begin{theorem}
$\sigma_\lambda$ is a field automorphism on $K_{a,n}$.
\end{theorem}
\begin{proof}
First since any $x \in K_{a,n}$
can be written uniquely as a sum
\begin{equation}
\label{uniqueExpression}
x = b_0 + b_1\sqrt[n]{a}+b_2\sqrt[n]{a^2} + \cdots + b_{n-1}\sqrt[n]{a^{n-1}},
\end{equation}
for $b_i \in k$, then $\sigma_\lambda(x)$ is given by
\begin{equation}
\label{uniqueExpression2}
\sigma_\lambda(x) = b_0 + b_1\sqrt[n]{a}\zeta_n^\lambda+b_2\sqrt[n]{a^2}\zeta_n^{2\lambda}+\cdots+b_{n-1}\sqrt[n]{a^{n-1}}\zeta_n^{\lambda(n-1)}.
\end{equation}
This shows that $\sigma_\lambda$ is well defined, since the expression in equation \ref{uniqueExpression}
is unique so $\sigma_\lambda$ must be given by \ref{uniqueExpression2}. 

To show that $\sigma_\lambda$ is bijective, it suffices to find an inverse function.
Let $x = f(\sqrt[n]{a}) \in K_{a,n}$, then
\begin{equation*}
    \sigma_{\lambda}(\sigma_{-\lambda}(x)) = f(\sqrt[n]{a}\zeta_n^{\lambda}\zeta_n^{-\lambda}) = x = \sigma_{-\lambda}(\sigma_\lambda(x)).
\end{equation*}

Suppose that $f(\sqrt[n]{a}),g(\sqrt[n]{a}) \in K_{a,n}$. Then,
\begin{align*}
\sigma_{\lambda}(f(\sqrt[n]{a})+g(\sqrt[n]{a}) &= \sigma_\lambda((f+g)(\sqrt[n]{a}))\\
&= (f+g)(\sqrt[n]{a}\zeta_n^\lambda)\\
&= f(\sqrt[n]{a}\zeta_n^\lambda) + g(\sqrt[n]{a}\zeta_n^\lambda)\\
&= \sigma_\lambda(f(\sqrt[n]{a}))+\sigma_\lambda(g(\sqrt[n]{a})).
\end{align*}
Hence the function $\sigma_\lambda$ is additive. An identical argument shows that $\sigma_\lambda$
is multiplicative.
%Since for $b \in k$, $\sigma_\lambda(bf(\sqrt[n]{a})) = b\sigma_\lambda(f(\sqrt[n]{a}))$,
%$\sigma_\lambda$ is well defined as it is defined uniquely on the expression given in 
%equation \ref{uniqueExpression}.



\end{proof}

\begin{corollary}
    For $\lambda \in \Intgr/n$, $\sigma_\lambda \in \Aut(K_{a,n}/k)$.
\end{corollary}
\begin{proof}
    Let $b \in k$, then we
    have $\sigma_\lambda(b) = b$ since $b$ is a degree $0$
    polynomial in $k[x]$.
    
    Hence $\sigma_\lambda$ fixes $k$.
\end{proof} 
\section*{Question 3}
\begin{lemma}
If $\sigma \in \Aut(K_{a,n}/k)$, then $P_a(\sigma(\sqrt[n]{a})) = 0$.
\end{lemma}
\begin{proof}
We simply compute $P_a(\sigma(\sqrt[n]{a}))$,
\begin{align*}
P(\sigma(\sqrt[n]{a})) &= \sigma(\sqrt[n]{a})^n-a\\
&= \sigma(\sqrt[n]{a}^n)-a\\
&= \sigma(a)-\sigma(a)\\
&= 0.
\end{align*}
\end{proof}
\begin{lemma}
If $\sigma \in \Aut(K_{a,n}/k)$, then $\sigma = \sigma_\lambda$ for some $\lambda \in \Intgr/n$.
\end{lemma}
\begin{proof}
We have shown that $\sigma(\sqrt[n]{a})$ is a root of $P_a$, however the only roots of $P_{a}$ over
$K_{a,n}$ are of the form $\zeta_{n}^\lambda \sqrt[n]{a}$ for some $\lambda \in \Intgr/n$.
Hence $\sigma(\sqrt[n]{a}) = \zeta_n^\lambda \sqrt[n]{a}$. Since $\{1,\sqrt[n]{a}\}$
generates $K_{a,n}$ as a $k$-space, this uniquely determines $\sigma$ as $\sigma_\lambda$.
\end{proof}
\begin{theorem}
Hence, $\Aut(K_{a,n}/k)$ is isomorphic to $\Intgr/n$ as a group.
\end{theorem}
\begin{proof}
The map $\lambda \mapsto \sigma_\lambda$ is a bijection since we
have shown that it is surjective, and if $\sigma_\lambda = \sigma_\mu$, then
$\zeta_n^\lambda = \zeta_n^\mu$ so $\lambda = \mu$ as $\zeta_n$ is primitive.
To show that this is a group homomorphism, let $\lambda,\mu \in \Intgr/n$.
Then $\sigma_\lambda\circ\sigma_\mu$ is a field automorphism fixing $k$,
and $\sigma_\lambda(\sigma_\mu(\sqrt[n]{a})) = \sqrt[n]{a}\zeta_n^\mu\zeta_n^\lambda = \sqrt[n]{a}\zeta_n^{\lambda+\mu}$.
Hence $\sigma_\lambda\circ \sigma_\mu = \sigma_{\lambda+\mu}$.
Hence we have an isomorphism of groups.
\end{proof}
\section*{Question 4}
For this question, $m|n$ is an integer, and $m(\Intgr/n)$ is the subgroup of $\Intgr/n$
generated by $m$.
\begin{lemma}
$\sqrt[m]{a}$ is fixed by $\sigma_\lambda$ for each $\lambda \in m(\Intgr/n)$.
\end{lemma}
\begin{proof}
Note that $\sqrt[m]{a} = \sqrt[n]{a}^{n/m}$ since $n/m$ is an integer.
Hence $\sigma_\lambda(\sqrt[m]{a}) = \sqrt[n]{a}^{n/m}\zeta_n^{n\lambda/m} = \sqrt[n]{a}\zeta_n^{n\lambda/m}$
Hence $\sigma_\lambda$ fixes $\sqrt[n]{a}$ if and only if $n| n\lambda/m$, so we must have $m|\lambda$.
Hence $\lambda \in m(\Intgr/n)$.
\end{proof}
\begin{theorem}
$\sigma_\lambda(u) = u$ for all $\lambda \in m(\Intgr/n)$ if and only if $u \in K_{a,m}$.
\end{theorem}
\begin{proof}
Suppose first that $u \in K_{a,m}$. It is sufficient to consider the case $u = \sqrt[m]{a}$
since $\sqrt[m]{a}$ generates $K_{a,m}$ as a $k$-algebra. We have already shown for this case that $\sigma_\lambda(u) = u$
for all $\lambda \in m(\Intgr/n)$.
Conversely, assume that $\sigma_\lambda(u) = u$ for each $\lambda \in m(\Intgr/n)$. We
know that $u$ can be uniquely written as
\begin{equation*}
u = b_0+b_1\sqrt[n]{a}+\cdots+b_{n-1}\sqrt[n]{a}^{n-1}.
\end{equation*}
for $b_i\in k$. Then if $\sigma_\lambda(u) = u$, we have
\begin{align*}
b_0+ b_1\sqrt[n]{a}\zeta_n^\lambda+\cdots+b_{n-1}\sqrt[n]{a}^{n-1}\zeta_n^{\lambda(n-1)} = b_0+b_1\sqrt[n]{a}+\cdots+b_{n-1}\sqrt[n]{a}^{n-1}.
\end{align*}
Hence by the uniqueness of this representation, we have $b_i = \zeta_n^{\lambda i}b_i$ for all $i = 0,\ldots,n-1$,
and for all $\lambda \in m(\Intgr/n)$. If $b_i \neq 0$, we must have $\zeta_n^{\lambda i} = 1$, in particular
$\zeta_n^{mi} = 1$. So $n|mi$. Hence there is some integer $p$ such that $pn/m = i$.

Thus, we can only have $b_i \neq 0$ when $i$ is a multiple of $n/m$. Hence each term 
in the expression of $u$ is a multiple of a power of $\sqrt[n]{a}^{n/m} = \sqrt[m]{a}$.

Thus, $u \in K_{a,m}$. 

If $u \in K_{a,m}$, we must prove that $\sigma_{\lambda}(u) = u$ for all $\lambda \in m(\Intgr/n)$.
However it is sufficent to consider $u = \sqrt[m]{a} = \sqrt[n]{a}^{n/m}$ since this
generates $K_{a,m}$ as a $k$-algebra. Clearly then $\sigma_\lambda(u) = \sqrt[m]{a}\zeta_n^{n\lambda/m}$.
However since $\lambda$ is a multiple of $m$, we conclude that $\zeta_n^{n\lambda/m} = 1$.
\end{proof}
\end{document}
