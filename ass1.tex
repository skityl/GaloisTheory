\documentclass{unswmaths}

\usepackage{unswshortcuts}

\begin{document}

\subject{Galois Theory}
\author{Edward McDonald}
\title{Assignment 1}
\studentno{3375335}


\setlength\parindent{0pt}


\unswtitle{}

    For this assignment, $n > 1$ and $k$ is a field
    containing a primitive $n$th root of unity $\zeta_n$,
    and the characteristic of $k$ does not divide $n$.
    
    Let $a \in k$ be such that the polynomial
    $P_a(x) = x^n-a$ has no root in $k$.
    
    Let $K_{a,n} = k(\!\sqrt[n]{a})$ be a field extention
    of $k$ generated by a root of $P_a$.

\section*{Question 1}
\begin{lemma}
    In $K_{a,n}$, $P_a$ is expressed as a product of $n$ distinct linear
    factors 
    \begin{equation*}
        P_a(x) = \prod_{k=1}^n (x-\zeta_n^k\sqrt[n]{a})
    \end{equation*}
\end{lemma}
\begin{proof}
    Since $\zeta_n$ is primitive, all
    of the elements $\zeta_n^k\sqrt[n]{a}$ for $k = 1,2,\ldots,n$
    are distinct and are all zeroes of $P_a$.
    
    Hence since $P_a$ has at most $n$ roots, this is all of the roots
    of $P_a$ and so $P_a$ decomposes as
    \begin{equation*}
        P_a(x) = \prod_{k=1}^n (x-\zeta_n^k\sqrt[n]{a})
    \end{equation*}
\end{proof}

\section*{Question 2}
For this question, $\lambda \in \Intgr/n\Intgr$ and 
define
\begin{equation*}
    \sigma_\lambda: K_{a,n}\rightarrow K_{a,n}
\end{equation*}
by $\sigma_{\lambda}(f(\sqrt[n]{a})) = f(\zeta_n^\lambda \sqrt[n]{a})$
for any polynomial $f \in k[x]$.
\begin{theorem}
    $\sigma_\lambda$ is a field automorphism on $K_{a,n}$.
\end{theorem}
\begin{proof}
    First since any $x \in K_{a,n}$
    can be written uniquely as a sum
    \begin{equation*}
        x = b_0 + b_1\sqrt[n]{a}+b_2\sqrt[n]{a^2} + \cdots + b_{n-1}\sqrt[n]{a^{n-1}},
    \end{equation*}
    for $b_i \in k$, then $\sigma_\lambda(x)$ is given by
    \begin{equation*}
        \sigma_\lambda(x) = b_0 + b_1\sqrt[n]{a}\zeta_n^\lambda+b_2\sqrt[n]{a^2}\zeta_n^{2\lambda}+\cdots+b_{n-1}\sqrt[n]{a^{n-1}}\zeta_n^{\lambda(n-1)}.
    \end{equation*}
    
    Suppose that $f(\sqrt[n]{a}),g(\sqrt[n]{a}) \in K_{a,n}$. Then,
    \begin{align*}
        \sigma_{\lambda}(f(\sqrt[n]{a})+g(\sqrt[n]{a}) &= \sigma_\lambda((f+g)(\sqrt[n]{a}))\\
        &= (f+g)(\sqrt[n]{a}\zeta_n^\lambda)\\
        &= f(\sqrt[n]{a}\zeta_n^\lambda) + g(\sqrt[n]{a}\zeta_n^\lambda)\\
        &= \sigma_\lambda(f(\sqrt[n]{a}))+\sigma_\lambda(g(\sqrt[n]{a})).
    \end{align*}
    Hence the function $\sigma_\lambda$ is additive. An identical argument shows that $\sigma_\lambda$
    is multiplicative.
    
    Since for $b \in k$, $\sigma_\lambda(bf(\sqrt[n]{a})) = b\sigma_\lambda(f(\sqrt[n]{a}))
    
\end{proof}

\section*{Question 3}


    
    
    
    
    

\end{document}